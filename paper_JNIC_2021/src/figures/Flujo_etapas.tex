
\begin{figure}[!ht]
  \centering
  % centering debería centrar adecuadamente la figura, así que, por norma general, no es necesario utilizar hspace
  %\hspace*{-135pt}
  \begin{tikzpicture}[node distance = 2cm, auto,
  % Estas dos opciones de tikzpicture son para poder reescalar la imagen
  transform shape, scale=0.8
  ]
    % nodes
    \node [blockWH] (prep_env) {\Large Prepare environment};
    \node [decisionBL, below of = prep_env, node distance = 3.5cm] (Hay_src) {\Large Is there source code available?};
    \node [blockBL, right of = Hay_src, node distance = 5cm] (prep_src) {\Large Prepare source code};
    \node [blockBL, below of = Hay_src, node distance = 3.5cm] (prep_bin) {\Large Prepare binary};
    \node [blockBL, below of = prep_bin, node distance = 2cm] (prep_in) {\Large Prepare inputs};
    \node [blockGR, below of = prep_in, node distance = 2cm] (pruebas) {\Large Fuzzing workload};
    \node [blockYL, below of = pruebas, node distance = 2cm] (resultados) {\Large Results analysis workloads};

    % edges
    \path [line] (prep_env) -- (Hay_src);
    \path [line] (Hay_src) -- node {Yes}(prep_src);
    \path [line] (Hay_src) -- node {\Large No}(prep_bin);
    \path [line] (prep_bin) -- (prep_in);
    \path [line] (prep_src) -- (prep_bin);
    \path [line] (prep_in) -- (pruebas);
    \path [line] (pruebas) -- (resultados);

  % Leyenda cuqui
  \matrix [draw,every cell/.style={scale=0.6}] at (5,-10) {
  \node [label=center:Phases/Steps] {}; \\
  \node [blockWH,label=center:Common] {}; \\
  \node [blockBL,label=center:Prefuzzing] {}; \\
  \node [blockGR,label=center:Fuzzing] {}; \\
  \node [blockYL,label=center:Postfuzzing] {}; \\
};

  \end{tikzpicture}
  \caption{ Flujo etapas fuzzing y actividades principales. }
  \label{fig:HOUSE_flow}
\end{figure}

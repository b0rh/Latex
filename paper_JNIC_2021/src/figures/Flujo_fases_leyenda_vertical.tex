
\begin{figure}[!ht]
  \centering
  % centering debería centrar adecuadamente la figura, así que, por norma general, no es necesario utilizar hspace
  %\hspace*{-135pt}
  \begin{tikzpicture}[node distance = 2cm, auto,
  % En un paper, las figuras suelen tener una fuente de tipo sin serifa (para
  % distinguirlas del texto). Además, todos elementos de texto deberían tener
  % el mismo tamaño de fuente)
  font=\large\sffamily,
  % Estas dos opciones de tikzpicture son para poder reescalar la imagen
  transform shape, scale=0.75
  ]
    % nodes
    \node [blockWH] (prep_env) {Preparar entorno de trabajo};
    \node [decisionBL, below of = prep_env, node distance = 3.5cm] (Hay_src) {¿Hay código fuente disponible?};
    \node [blockBL, right of = Hay_src, node distance = 5cm] (prep_src) {Preparar código fuente};
    \node [blockBL, below of = Hay_src, node distance = 3.5cm] (prep_bin) {Preparar binarios};
    \node [blockBL, below of = prep_bin, node distance = 2cm] (prep_in) {Preparar entradas};
    \node [blockGR, below of = prep_in, node distance = 2cm] (pruebas) {Paquete de trabajo de fuzzing};
    \node [blockYL, below of = pruebas, node distance = 2.3cm] (resultados) {Analizar resultados de los paquetes de trabajo};

    % edges
    \path [line] (prep_env) -- (Hay_src);
    \path [line] (Hay_src) -- node[midway,above] {Sí} (prep_src);
    \path [line] (Hay_src) -- node[midway,right] {No}(prep_bin);
    \path [line] (prep_bin) -- (prep_in);
    \path [line] (prep_src) |- (prep_bin);
    \path [line] (prep_in) -- (pruebas);
    \path [line] (pruebas) -- (resultados);

  % Leyenda cuqui
  \node [left of={prep_env},xshift=-2cm,blockWH,rotate=90,text width=2.2cm] (center:Common) {Común};
  \node [left of={prep_bin},xshift=-2cm,yshift=1.32cm,blockBL,rotate=90,text width=8.5cm] (center:Prefuzzing) {Prefuzzing};
  \node [left of={pruebas},xshift=-2cm,yshift=-0.11cm,blockGR,rotate=90,text width=1.8cm] (center:Fuzzing) {Fuzzing};
  \node [left of={resultados},xshift=-2cm,yshift=-0.1cm,blockYL,rotate=90,text width=2.2cm] (center:Postfuzzing) {Postfuzzing};

  %\node [above of={center:Common}] {Phases/Steps};

  \end{tikzpicture}
  \caption{ Diagrama de flujo para describir el funcionamiento de HOUSE según las diferentes fases de fuzzing. }
  \label{fig:HOUSE_flow}
\end{figure}

%%% PAQUETES  %%%

%%% Soporte extendido de colores
\usepackage{xcolor}
    % Definiciones colores extra
    %% Ref: http://latexcolor.com/
    \definecolor{coolblack}{rgb}{0.0, 0.18, 0.39}
    \definecolor{dimgray}{rgb}{0.41, 0.41, 0.41}

%%% Paquete para poner hipervinculos más finos \href
\PassOptionsToPackage{hyphens}{url}
\usepackage{hyperref}

   %Colores personalizados
    \hypersetup{
        colorlinks=true,
        linkcolor=black,
        filecolor=black,
        %%% Necesita su propia definicion de color
        citecolor=dimgray,
        urlcolor=coolblack,
    }

%%% Formato y tipografía de URL, direcciones de correo...
\usepackage{url}
    \def\UrlFont{\rmfamily}

%%% Usados por graficas de barras
\usepackage{textcomp}
\usepackage{tikz}
\usepackage{pgfplots}
\usepackage{pgfplotstable} 
\usepackage{csvsimple}% Generates table from .csv
\pgfplotsset{compat=1.7}
\usepackage{subcaption}
\usepgfplotslibrary{groupplots}

% PGFPlots Settings
\pgfplotsset{
SmallBarPlot/.style={
    font=\footnotesize,
    ybar,
    width=\linewidth,
    ymin=0,
    xtick=data,
    xticklabel style={text width=1.5cm, rotate=90, align=center}
},
BlueBars/.style={fill=blue!20, bar width=0.25},
RedBars/.style={fill=red!20, bar width=0.25},
GreenBars/.style={fill=green!20, bar width=0.25}
}

%% Para incluir en el texto o en el pie de la figura la leyenda con colores.
\DeclareRobustCommand\legendbox[1]{(\textcolor{#1}{#1 bars}~\begin{tikzpicture}[x=0.2cm, y=0.2cm] \draw [color=black, fill=#1!20] (0,0) -- (0,1) -- (0.6,1) -- (0.6,0) -- (0, 0); \end{tikzpicture})}

\usepackage{booktabs} % usado en tablas generadas con JASP


%%%%%%%%%%%%%%%%%%%%%%%%%%%%%%% HEREDADOS JNIC %%%%%%%%%%%%%%%%%%%%%%%%%%%%%%%

%%% TODO NOTES y definición 2cm de margen para que no se solapen añadido colores comentarios
 \setlength {\marginparwidth }{2cm} 
 \usepackage{todonotes}
 \usepackage[normalem]{ulem}

%%% Paquete para comentarios
\usepackage{verbatim}

%%% Tildes y demás caracteres en castellano...
%\usepackage[latin1]{inputenc}
% o bien
\usepackage[utf8]{inputenc}

%%% Fuente Times...
\usepackage{times}

%%% Figuras en formato .png, .ps, pdf o eps
\usepackage{graphicx}
%%\usepackage{subfigure}
\DeclareGraphicsExtensions{.png,.eps,.ps,.pdf}

%%% Soporte graficos svg
\usepackage{svg}

%%% Sección para definir explícitamente la separación de sílabas al final de una línea:
\hyphenation{si-guien-do}

%%% Secciones etc. en castellano
%\usepackage[spanish,es-tabla]{babel}

%%% Secciones etc. en Ingles
\usepackage[english]{babel}


%%% Paquete para usar simbolos y scripts de dibujado
\usepackage{tikz}
\def\checkmark{\tikz\fill[scale=0.4](0,.35) -- (.25,0) -- (1,.7) -- (.25,.15) -- cycle;} 
\usetikzlibrary{shapes,arrows}

% Define Block styles
\tikzstyle{decisionBL} = [diamond, draw, fill=blue!20,text width=10em, text badly centered, node distance=3cm, inner sep=0pt]
\tikzstyle{blockBL} = [rectangle, draw, fill=blue!20,text width=9em, text centered, rounded corners, minimum height=4em]
\tikzstyle{blockYL} = [rectangle, draw, fill=yellow!20, text width=9em, text centered, rounded corners, minimum height=4em]
\tikzstyle{blockGR} = [rectangle, draw, fill=green!20, text width=9em, text centered, rounded corners, minimum height=4em]
\tikzstyle{blockRD} = [rectangle, draw, fill=red!20, text width=9em, text centered, rounded corners, minimum height=4em]
\tikzstyle{blockWH} = [rectangle, draw, fill=white!20, text width=9em, text centered, rounded corners, minimum height=4em]
\tikzstyle{line} = [draw, -latex']
\tikzstyle{cloud} = [draw, ellipse,fill=red!20, node distance=4.5cm,minimum height=4em]

%%% Para usar ficheros tex de infografias Inkscape
\usepackage{pstricks}
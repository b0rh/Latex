
\begin{figure}[!ht]
  \centering
  % centering debería centrar adecuadamente la figura, así que, por norma general, no es necesario utilizar hspace
  %\hspace*{-135pt}
  \begin{tikzpicture}[node distance = 2cm, auto,
  % En un paper, las figuras suelen tener una fuente de tipo sin serifa (para
  % distinguirlas del texto). Además, todos elementos de texto deberían tener
  % el mismo tamaño de fuente)
  font=\large\sffamily,
  % Estas dos opciones de tikzpicture son para poder reescalar la imagen
  transform shape, scale=0.45
  ]
    % nodes
    \node [blockWH] (prep_env) {Preparar entorno de trabajo};
    \node [decisionBL, below of = prep_env, node distance = 3.5cm] (Hay_src) {¿Hay código fuente disponible?};
  

    % edges
    \path [line] (prep_env) -- (Hay_src);

  % Leyenda cuqui
  \node [left of={prep_env},xshift=-2cm,blockWH,rotate=90,text width=2.2cm] (center:Common) {Común};


  %\node [above of={center:Common}] {Phases/Steps};

  \end{tikzpicture}
  %\caption{ Diagrama de flujo para describir el funcionamiento de HOUSE según las diferentes fases de fuzzing. }
  \label{fig:HOUSE_flow}
\end{figure}
